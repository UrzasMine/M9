
\section{Zusammenfassung}
Die zentrale Erkenntnis des Versuchs, die Abhängigkeit der Phasengeschwindigkeit, von stehenden Wellen, von der Spannkraft, wurde bestätigt. Die graphische Methode sowie die Methode, welche den Mittelwert verwendet, kommen beide zu einer sehr ähnlichen Phasengeschwindigkeit bei konstanter Spannkraft. Bei der Bestimmung der Massenbelegung wurden ebenfalls zwei verschiedene Methoden angewandt. Hier haben sich die Ergebnisse deutlich unterschieden. Dies ist auf die Subjektivität zurückzuführen, die vorliegt, wenn man den Knotenpunkt auf die Aufhängung legen will.
Trotz der vorhandenen Messunsicherheiten stimmen die experimentell bestimmten Werte insgesamt mit der theoretisch erwarteten Abhängigkeit überein. Der Versuch zeigt damit qualitativ wie auch quantitativ den Zusammenhang zwischen Spannkraft, Massenbelegung und Phasengeschwindigkeit. Zur weiteren Verbesserung der Genauigkeit wäre ein modifizierter Versuchsaufbau oder eine objektivere Bestimmung der Knotenposition sinnvoll. Welche aber vermutlich einen anderen Versuchsaufbau benötigt.