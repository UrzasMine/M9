\section{Auswertung}
\subsection{Phasengeschwindigkeit bei konstanter Kraft}
Aus Gleichung \ref{lambdaf} lassen sich die Phasengeschwindigkeiten ausrechnen für eine Kraft von $(0,05 \pm 0,03)\,\text{N}$. Der Fehler der Kraft entsteht hierbei durch den Ablesefehler der Anzeige sowie dadurch, dass der Kraftmesser in einem Bereich von $\pm 0,03$ um das Messergebnis geschwankt ist. Da bei Aufgabe 2 die Messung von Aufgabe 1 wiederholt wurde, hier aber besser gemessen wurde, werden die Werte der zweiten Messung verwendet. Diese sind in Tabelle \ref{tab:F05} zu sehen.

\begin{table}[H]
    \centering
    \begin{tabular}{ccc}
        \toprule
        $\lambda/2$ (cm) & $f$ (Hz) & $c \left(\frac{\text{m}}{\text{s}} \right)$ \\
        \midrule
        69,0 & 14,1 & 19,45\\
        34,5 & 30,3 & 20,90\\
        22,5 & 44,8 & 20,16\\
        17,5 & 58,6 & 20,51\\
        13,8 & 73,3 & 20,23\\
        \bottomrule
    \end{tabular}
    \caption{Messwerte für die Wellenlänge mit zugehörigen Frequenzen und deren Produkt, die Phasengeschwindigkeit, für die Zugkraft $F = 0{,}51$\,N}
    \label{tab:F05}
\end{table}

Der Mittelwert der Phasengeschwindigkeit ist $\bar{c} = 20{,}25\left(\frac{\text{m}}{\text{s}} \right)$.

Die Standardabweichung ist durch
\begin{equation}
\sigma = \sqrt{\frac{1}{5 \cdot (5 - 1)} \sum_{i=1}^{5} \left(c_i - \bar{c}\right)^2}
\end{equation}

gegeben. Es ergibt sich somit für die Phasengeschwindigkeit für eine Spannkraft von $F = 0{,}51$\,N
\begin{equation}
    \bar{c} = (20{,}25 \pm 0,24)\frac{\text{m}}{\text{s}}
\end{equation}

Zudem kann die Phasengeschwindigkeit auch graphisch bestimmt werden. Mit Hilfe von Gleichung \ref{Resonanz} ergibt sich für die Steigung $m$
\begin{equation}
m = \frac{c}{2L}~{.}
\end{equation}

Somit erhält man für die Phasengeschwindigkeit $c$
\begin{equation}
c = 2 \cdot m \cdot L
\end{equation}

Trägt man nun die Frequenzen aus Tabelle \ref{tab:F05} gegen die Ordnung $n$ auf, so erhält man Abbildung \ref{nf}.

\begin{figure}[H]
    \centering
    \includegraphics[width=\textwidth]{Bilder/n gegen f.png}
    \caption{Ordnung $n$ aufgetragen gegen die Frequenz $f$ für eine Spannkraft von $0{,}51$\,N. Die Ausgleichs- und Grenzgeraden wurden mittels Python erstellt.}
    \label{nf}
\end{figure}

Mit den Steigungen aus Abbildung \ref{nf} ergibt sich ein Fehler für $m$ von

\begin{equation}
= \pm \left| \frac{m_1 - m_2}{2} \right|
= \pm \left| \frac{15{,}46\,\mathrm{Hz} - 13{,}93\,\mathrm{Hz}}{2} \right|
\approx 0{,}77 \,\mathrm{Hz}.
\end{equation}

Und für die Phasengeschwindigkeit

\begin{equation}
c = 2 \cdot 14{,}67~\text{Hz} \cdot 0{,}69~\text{m} \approx 20{,}24 \frac{\text{m}}{\text{s}}.
\end{equation}

Der Fehler ist durch
\begin{equation}
\Delta c
= \pm \left(
\left| \frac{\partial c}{\partial m} \right| \Delta m
+
\left| \frac{\partial c}{\partial L} \right| \Delta L
\right)
= \pm 2 \cdot \left( m \cdot \Delta L + L \cdot \Delta m \right)\approx 0{,}58 \frac{\text{m}}{\text{s}}
\end{equation}
gegeben.

In Tabelle \ref{ver} sind die Ergebnisse aus den zwei Verfahren nochmals aufgelistet.

\begin{table}[H]
    \centering
    \begin{tabular}{ccc}
        \toprule
        Verfahren & $c~\left(\frac{\text{m}}{\text{s}}\right)$ & $\Delta c~\left(\frac{\text{m}}{\text{s}}\right)$ \\
        \midrule
        Mittelwert & 20{,}25 & 0,24\\
        Grafisch & 20{,}24 & 0{,}58\\
        \bottomrule
    \end{tabular}
    \caption{Die Phasengeschwindigkeiten aus den verschiedenen Verfahren mit Fehler}
    \label{ver}
\end{table}

Die beiden Ergebnisse liegen in derselben Größenordnung und stimmen gut miteinander überein. Die mittels Mittelwert bestimmten Ergebnisse liegen innerhalb des Bereichs der graphischen Auswertung, wohingegen dies umgekehrt nicht der Fall ist. 

Der Mittelwert weist einen deutlich kleineren absoluten auf und erscheint daher als die genauer. Mit einem relativen Fehler von 2{,}9\,\% liefert jedoch auch die graphische Auswertung einen validen Wert für die Phasengeschwindigkeit der stehenden Welle.
\\

Außerdem soll die Massenbelegung des Gummiseils berechnet werden. Diese ist durch Gleichung \ref{Massenbelegung} gegeben. Mit einer Masse des Gummiseils von $m = 1{,}1$\,g und einer Länge von $66{,}5$\,cm $\pm 0{,}3$\,cm. Daraus ergibt sich eine Massenbelegung von
\begin{equation}
\mu = (1{,}79 \cdot 10^{-4} \pm 0{,}74) \frac{\text{g}}{\text{m}}
\end{equation}

Der Fehler ist dabei durch
\begin{equation}
\Delta \mu = \pm \left(
\left| \frac{\partial \mu}{\partial m} \right| \Delta m
+
\left| \frac{\partial \mu}{\partial L} \right| \Delta L
\right)
= \pm \left(
\frac{\Delta m}{L}
+
\frac{m}{L^2}\,\Delta L
\right)
\end{equation}

gegeben. Wobei die Masse des Gummiseils als fehlerlos angenommen wurde.

\subsection{Abhängigkeit Phasengeschwindigkeit-Massenverteilung}

Zuerst werden die gemessenen Phasengeschwindigkeiten für die jeweiligen Spannkräfte nach Gleichung \ref{lambdaf} berechnet. Die über alle Ordnungen gemittelten Werte sind in Tabelle \ref{tab:cs} abgebildet. 

\begin{table}[H]
\centering
\begin{tabular}{c c c}
\hline
$F$ Spannkraft & Gemittelte $\bar c$ & $\sigma$\\
\hline
0.109 & 11.563 & 1.166 \\
0.210 & 12.554 & 1.023 \\
0.300 & 16.074 & 0.561 \\
0.439 & 18.157 & 0.542 \\
0.510 & 20.253 & 0.533 \\
0.600 & 21.817 & 0.471 \\
0.690 & 23.613 & 0.311 \\
0.810 & 25.040 & 0.664 \\
0.900 & 27.459 & 0.573 \\
1.000 & 28.810 & 0.340 \\
\hline
\end{tabular}
\caption{Spannkraft $F$ und die jeweiligen gemittelten Phasengeschwindigkeiten $c$ mit der Standartabweichung $\mu$}
\label{tab:cs}
\end{table}

Wie Gleichung \ref{Phasengeschwindigkeit} erwarten lässt, steigt die Phasengeschwindigkeit mit der Spannkraft. Trägt man diese Werte in ein Grafik auf, so erhält man Abbildung \ref{inding}. 

\begin{figure}[H]
    \centering
    \includegraphics[width=\textwidth]{Bilder/lndingens.png}
    \caption{Die gemittelten Phasengeschwindigkeiten aufgetragen gegen die jeweilige Spannkraft. Die Fehlerbalken stellen die Standardabweichung dar. Für die Ausgleichsfunktion wurde ein Wurzelansatz gewählt.}
    \label{inding}
\end{figure}

Die Werte lassen sich durch eine Wurzelfunktion mäßig gut nähern. Zudem sind die Standardabweichungen bei niedrigen Spannkräften recht hoch. Dies hat den Grund, dass bei geringeren Spannkräften es schwieriger war, den äußeren Knotenpunkt genau an die Aufhängung bei dem Frequenzgenerator zu legen. Wird die quadratische Phasengeschwindigkeit $\bar c^2$ gegen die Spannkraft $F$ aufgetragen, so erhält man Abbildung \ref{quadrat}.

\begin{figure}[H]
    \centering
    \includegraphics[width=\textwidth]{Bilder/c2.png}
    \caption{Die quadratischen gemittelten Phasengeschwindigkeiten aufgetragen gegen die jeweilige Spannkraft. Die Fehlerbalken stellen die Standardabweichung dar. Die Ausgleichs- und Grenzgeraden wurden mittels Python.numpy erstellt.}
    \label{quadrat}
\end{figure}

Wie erwartet ist ein linearer Zusammenhang zu erkennen. Nun kann mit der Inversen der Steigung gemäß Gleichung \ref{Phasengeschwindigkeit}
\begin{equation}
m = \frac{c^2}{F}
\end{equation}

die Massenbelegung $\mu$ berechnet werden. Die Steigungen aus Abbildung \ref{quadrat} sind in Tabelle \ref{tab:m} aufgelistet.

\begin{table}[H]
\centering
\begin{tabular}{c c}
\hline
Gerade & Steigung $\left( m \right)$\\
\hline
Ausgleichsgerade & 800.89 \\
Grenzgerade 1 & 694.44 \\
Grenzgerade 2 & 923.63 \\
\hline
\end{tabular}
\caption{Steigungen der Geraden aus Abbildung \ref{quadrat}}
\label{tab:m}
\end{table}

Es ergibt sich mit der Steigung der Ausgleichsgerade eine Massenverteilung von
\begin{equation}
\mu = (1,2 \pm 1,3) \frac{g}{m}
\end{equation}

wobei der Fehler durch
\begin{equation}
\Delta m = \pm \left| \frac{m_1 - m_2}{2} \right|
\end{equation}

gegeben ist. Das Ergebnis ist hier mit einem Fehler von mehr als 100\,\% sehr unzuverlässig. Das hier errechnete Ergebnis weicht von dem oben errechneten Wert um
\begin{equation}
\frac{\mu_2}{\mu_1} = 50 \%
\end{equation}

ab. Damit sind sie zwar in der gleichen Größenordnung, weichen aber trotzdem erheblich voneinander ab. Insgesamt scheint das oben errechnete Ergebnis zuverlässiger, da hier zum einen der Größtfehler geringer ist, zum anderen das Ergebnis oben von genauer zu bestimmenden Größen abhängt. Es musste hier nur die Länge des Gummiseils im Ruhezustand gemessen werden, die Masse des Gummikabels war in der Versuchsanleitung gegeben. Bei den Punkten im Graphen gibt es einige Fehlerquellen, der größte ist vermutlich, dass die Wellenlänge der stehenden Wellen vor allem bei niedrigen Spannkräften sehr ungenau ist. Dies ist auch in Abbildung \ref{quadrat} und in Abbildung \ref{inding} zu erkennen.

Grundsätzlich erweist es sich in diesem Versuch als schwierig, die exakte Frequenz für die jeweilige Eigenschwingung einzustellen. Zwar wurden Kriterien zur Festlegung des gewünschten Schwingungsbildes eingehalten, wie beispielsweise die Symmetrie der Bäuche, die Position und die Abstände der Knoten sowie das Annähern der Frequenz von unten und sekundär die größte Amplitude. Dennoch unterliegt die Beurteilung stets dem subjektiven Eindruck des Beobachters, wodurch eine gewisse Subjektivität und damit eine Abweichung der Messwerte vom tatsächlichen Wert entsteht.

