\section{Auswertung}
\subsection{Phasengeschwindigkeit bei Konstanter Kraft}
Aus Gleichung \ref{lambdaf} lassen sich die Phasengeschwindigkeiten Ausrechen Für eine Kraft von $(0,05 \pm 0,03) \text{N}$. Der Fehler der Kraft entseht hierei durch den Ablesefehler der Anzeige, sowie davon, dass der Kraftmesser in einem Bereich von $\pm 0,03$ um das messergebniss geschwankt ist. Da bei Aufgabe 2 Die Messung von Aufgabe 1 wiederholt wurde, hier aber Besser gemmessen wurde, werden die Werte von der zweiten Messung verwendet. Diese sind in Tabelle \ref{tab:F05} zu sehen.

\begin{table}[H]
    \centering
    \begin{tabular}{ccc}
        \toprule
        $\lambda/2$ (cm) & $f$  (Hz) & $c \left(\frac{\text{m}}{\text{s}} \right)$ \\
        \midrule
        69,0 & 14,1 & 19,45\\
        34,5 & 30,3 & 20,90\\
        22,5 & 44,8 & 20,16\\
        17,5 & 58,6 & 20,51\\
        13,8 & 73,3 & 20,23\\
        \bottomrule
    \end{tabular}
    \caption{Messwerte für die Wellenlänge mit zugehörigen Frequenzen und deren Produkt die Phasengeschwidigkeit für die Zugkraft $F = 0{,}51$N}
    \label{tab:F05}
\end{table}

Der Mittelwert der Phasengeschwindikeit ist  $\bar{c} = 2025,32\left(\frac{\text{m}}{\text{s}} \right)$ 

Die Standartabweichung ist durch
\begin{equation}
\sigma = \sqrt{\frac{1}{5 \cdot (5 - 1)} \sum_{i=1}^{5} \left(c_i - \bar{c}\right)^2}
\end{equation}

gegeben. Es ergibt sich somit für Die Phasengeschwindikeit für eine Spannkraft von $F = 0{,}51$N
\begin{equation}
    \bar{c} = (20,25 \pm )\frac{\text{m}}{\text{s}}
\end{equation}

Zudem kann die phasen geschwindigkeit auch graphisch bestimmt werden. Mit Hilfe von Gleichung \ref{} ergibt sich für die Steigung $m$
\begin{equation}
m = \frac{c}{2L}~{.}
\end{equation}

Somit erhählt man fpr die Phasengeschwinigkeit $c$
\begin{equation}
c = 2 \cdot m \cdot \cdot L 
\end{equation}

Trägt man nun die frequenzen aus Tabelle \ref{tab:F05} gegen die Ordnung $n$ auf so erhält man Abbildung \ref{nf}

\begin{figure}[H]
    \centering
    \includegraphics[width=\textwidth]{Bilder/n gegen f.png}
    \caption{Ordung $n$ aufgetragen gegen die Frequenz $f$ für eine Spannkraft von $0,51$N Die Ausgleichs und grenzgeraden wurden mittels Phyton erstellt.}
    \label{nf}
\end{figure}

Mit den Steigungen aus Abbildung \ref{nf} ergibt sich ein Fehler für $m$ von 

\begin{equation}
= \pm \left| \frac{m_1 - m_2}{2} \right|
= \pm \left| \frac{15{,}46\,\mathrm{Hz} - 13{,}93\,\mathrm{Hz}}{2} \right|
\approx 0{,}77  \,\mathrm{Hz} .
\end{equation}

Und für die Phasengeschwindigeit

\begin{equation}
c = 2 \cdot 14,67~\text{Hz} \cdot 0,69~\text{m} \approx 20,24 \frac{\text{m}}{\text{s}} .
\end{equation}

Der Fehler ist durch 
\begin{equation}
\Delta c
= \pm \left(
\left| \frac{\partial c}{\partial m} \right| \Delta m
+
\left| \frac{\partial c}{\partial L} \right| \Delta L
\right)
= \pm 2 \cdot \left( m \cdot \Delta L + L \cdot \Delta m \right)\approx 0,58 \frac{\text{m}}{\text{s}}
\end{equation} 
gegeben.

In Tabelle \ref{ver} sind die ergebnisse Aus den Zwei verfahren nochmals aufgelistet 

\begin{table}[H]
    \centering
    \begin{tabular}{ccc}
        \toprule
        Verfahren & $c \left(\frac{\text{m}}{\text{s}}\right)$ &$\Delta c\left(\frac{\text{m}}{\text{s}}\right)$  (Hz)\\
        \midrule
        Mittelwert & 20,25 & FEHLER\\
        Grafisch & 20,24 & 0,58\\

        \bottomrule
    \end{tabular}
    \caption{Die Phasengeschwindigkeiten aus den Verschiedenen Verfahren mit Fehler}
    \label{ver}
\end{table}

Ausserdem soll die Massenbelegung des Gummiseils berechent werden. 
\dots

\subsection{Abhängigkeit Phasengeschwidigkeit-Massenverteilung}

Zuerst werden die gemessene Phasengeschwindigkeiten für die jeweiligen Spannkräfte nach Gleichung \ref{lambdaf} berechnet. Die über alle ordnungen gemittelten Werte sind in Tablle \ref{tab:cs} abgebildet. 


\begin{table}[H]
\centering

\begin{tabular}{c c c}
\hline
$F$ Spannkaft & Gemittelte $\bar c$ & $\sigma$\\
\hline
0.109 & 578.154 & 58.319 \\
0.210 & 627.709 & 51.150 \\
0.300 & 803.724 & 28.051 \\
0.439 & 907.828 & 27.076 \\
0.510 & 1012.658 & 26.633 \\
0.600 & 1090.864 & 23.556 \\
0.690 & 1180.644 & 15.562 \\
0.810 & 1252.010 & 33.176 \\
0.900 & 1290.716 & 51.575 \\
1.000 & 1454.150 & 97.630 \\
\hline
\end{tabular}
\caption{Spannkraft $F$ und die jeweiligen gemittelten Phasengeschwindigkeiten $c$}
\label{tab:cs}
\end{table}

Trägt man diese Werte in ein Grafik auf so erhält man Abbildung \ref{inding}


\begin{figure}[H]
    \centering
    \includegraphics[width=\textwidth]{Bilder/lndingens.png}
    \caption{Die gemittelten Phasengeschwindigkeiten aufgetragen gegen die jeweilige Spannkraft. Die Fehlerbalken stellen die Standartabweichung dar. }
    \label{inding}
\end{figure}