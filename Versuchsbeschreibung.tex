\section{Versuchsbeschreibung}

\subsection{Versuchsaufbau}
\begin{figure}[H]
    \centering
    \includegraphics[width=\textwidth]{Bilder/Aufbau M9.jpg}
    \label{Aufbau}
    \caption{Eine selbst angefertigte Skizze zum Versuchsaufbau}
\end{figure}

Für den Versuchsaufbau werden benötigt: ein Gummiseil mit bekannter Masse, ein verstellbarer Funktionsgenerator, ein Schwingungsgenerator, ein Kraftmesser mit der Möglichkeit, diesen Wert auszulesen, sowie ein Maßband. Für den Aufbau wird der Kraftmesser an einer Stange befestigt, damit er vom Tisch entfernt ist, und mit dem Auslesegerät verbunden. Daraufhin wird das Gummiseil an der einen Seite am Kraftmesser befestigt und auf der anderen Seite am Schwingungsgenerator. Als nächstes wird der Schwingungsgenerator am Funktionsgenerator angeschlossen, sodass harmonische Schwingungen erzeugt werden können. Um die Durchführung zu erleichtern, sollte entweder der Schwingungsgenerator oder der Kraftmesser verschiebbar sein. Wenn der Aufbau nun vollendet ist, sollte er ungefähr so aussehen wie in Abbildung \ref{Aufbau}.

\subsection{Versuchsdurchführung}
Für die Durchführung des Versuches wird zu Beginn, ohne Spannung auf dem Gummiseil, der Kraftmesser kalibriert. Daraufhin wird das Gummiseil so lange gespannt, bis der Kraftmesser 0{,}5 Newton anzeigt. Nun wird am Funktionsgenerator eine Amplitude von 5~V eingestellt und die Frequenz langsam erhöht, bis der erste Wellenbauch entsteht, also $n = 1$. Wenn dieser entstanden ist, werden die Frequenz und $\frac{\lambda}{2}$ aufgezeichnet. Dies wird nun für alle $n$ bis 5 wiederholt. Hierbei ist darauf zu achten, dass die Frequenz 100~Hz nicht übersteigt. Sobald diese Messung beendet wurde, wird zusätzlich die Kraft in 0{,}1-Newton-Schritten von 0{,}1 bis 1~Newton variiert und erneut die Resonanzfrequenzen bis $n = 5$ untersucht.
